%! Suppress = Unicode
%! Author = Adrian Helberg
%! Script = 18.02.2020
%! Presentation = 21.02.2020

%%% Preamble
\documentclass[11pt]{article}

%%% Packages
\usepackage{ngerman}
\usepackage{color} % Colored font
\usepackage{graphicx} % Images
\usepackage{titling} % Title formatting
\usepackage[utf8]{inputenc} % UTF-8
\usepackage {tikz} % Graphs

%%% Commands
% ToC title
\renewcommand{\contentsname}{
    \begin{center}
        \LARGE Inhalt
    \end{center}
}

%%% Title
\pretitle{
    \begin{center}
        \includegraphics[width=8cm]{../../resources/Haw_logo.png}
    \end{center}
}
\title{
    \begin{center}
        \Huge \textbf{Intelligente Systeme}\\
        {\color{blue}- Skript -}\\~\\
        \Large Prüfungsform: Referat
    \end{center}
}
\author{Adrian Helberg\\ Matr.Nr. 2309051}
\date{Abgabe: 18.02.2020}

%%% Document
\begin{document}

    \maketitle

    \begin{abstract}
        Diese Arbeit beschäftigt sich mit Konzepten der vier Kernbereiche aus \textit{intelligenten
        Systemen}: \textbf{Suchen}, \textbf{Lernen}, \textbf{Verarbeitung von Sequenzen} und
        \textbf{Ethik}. Zum einen werden Problemstellungen aufgezeigt, zum anderen auf
        Implementationen verwiesen, die in der zugehörigen Präsentation vorgestellt werden.\\ Eine
        Lösungsstrategie für das \textit{Problem des Handlungsreisenden (Traveling Salesman
        Problem)} wird im Kapitel \textbf{Suchen} beschrieben. \textit{Selbstorganisierte Karten
            (Selforganizing Maps)} zeigen eine Strategie des \textbf{Lernen}s, ein \textit{Deep
        Learning}- Ansatz zur Datenanalyse von Sensordaten mobiler Endgeräte gibt Einblicke in
        die \textbf{Sequenzverarbeitung} und eine Diskussion zum \textit{Trolley Problem} lädt
        zur Diskussion über die \textbf{Ethik} in Verbindung mit \textit{intelligenten Systemen}
        ein.
    \end{abstract}

    \tableofcontents
    \newpage

    \section{Suchen}

    Dieses Kapitel beschäftigt sich mit der Umsetzungen einer Lösungsstrategie für das
    \textit{Problem des Handlungsreisenden (Traveling Salesman Problem)} mithilfe einer
    informierten Suche. Im speziellen eine Umsetzung mit einem genetischen Algorithmus.

    \subsection{Problem des Handlungsreisenden}
    Das Problem des Handlungsreisenden ist ein kombinatorisches Optimierungsproblem der
    theoretischen Informatik. Dabei muss ein Handlungsreisender eine Menge von Städten besuchen.
    Er beginnt bei einer bestimmten Stadt und muss, nachdem jede andere Stadt besucht wurde, zu
    dieser zurückkehren. Das Optimierungsproblem besteht bei der Festlegung der Reihenfolge der zu
    besuchenden Städte, sodass die gesamte Distanz der Reise minimal ist. Das Problem ist als
    NP-vollständig klassifiziert.

    \subsection{Genetischer Algorithmus}
    Evolutionäre Algorithmen sind eine Klasse von stochastischen, heuristischen
    Optimierungsverfahren. Der Name lässt sich von der Evolution natürlicher Lebewesen ableiten.
    Ziel von genetischen Algorithmen ist es optimierte Lösungen zu Aufgabenstellungen zu finden,
    bei denen ein Auffinden einer akzeptablen Lösung aus Gründen der kombinatorischen Komplexität
    misslingt. So gibt es beim Problem des Handlungsreisenden mit 10 Orten z.B. bereits
    $10!=3628800$ Lösungen. Kern eines genetischen Ansatzes ist die Veränderung von Mengen an
    Problemlösungen, sodass gute Lösungen mit einer großen Wahrscheinlichkeit und schlechte
    Lösungen mit geringen Wahrscheinlichkeit erhalten bleiben. Das Zusammenführen von Teilen guter
    Lösungen kann noch bessere Ergebnisse liefern. So kann verhindert werden, dass ein
    Algorithmus zur Optimierung an einem lokalen Optimum "hängenbleibt".

    \newpage

    \subsection{Allgemeiner Ablauf}
    \\~~\\
    \begin{center}
        \begin{tikzpicture}[semithick , state/.style ={ rectangle ,top color =white , bottom color = processblue!20 ,
        draw,processblue , text=blue , minimum width =1 cm}]
            \node[state] (A) at (0,0) {Start};
            \node[state] (B) at (0,-1.2) {Erzeugung einer zufälliger Population};
            \node[state] (C) at (0,-2.4) {Wiederhole bis Abbruchbedungung erfüllt};
            \node[state] (D) at (0,-3.6) {Berechne Fitness};
            \node[state] (E) at (0,-4.8) {Selektion};
            \node[state] (F) at (0,-6) {Crossover};
            \node[state] (G) at (0,-7.2) {Mutation};
            \node[state] (H) at (0,-8.4) {Austausch};
            \node[state] (I) at (0,-9.6) {Teste Abbruchbedingung};
            \node[state] (J) at (0,-10.8) {Ende};

            \draw[->, very thick] (A) to (B);
            \draw[->, very thick] (B) to (C);
            \draw[->, very thick] (C) to (D);
            \draw[->, very thick] (D) to (E);
            \draw[->, very thick] (E) to (F);
            \draw[->, very thick] (F) to (G);
            \draw[->, very thick] (G) to (H);
            \draw[->, very thick] (H) to (I);
            \draw[->, very thick] (I) to (J);
        \end{tikzpicture}
    \end{center}

    \subsection{Realisierung mit Java}

    \section{Lernen}


    \section{Sequenzen}


    \section{Ethik}


    \section{Quellen}

\end{document}