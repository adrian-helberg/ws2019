%! Author = Adrian Helberg
%! Abgabe = 18.02.2020

% Preamble
\documentclass[11pt]{article}

%%% Packages
\usepackage{ngerman}
\usepackage{color} % Colored font
\usepackage{graphicx} % Images
\usepackage{titling} % Title formatting
\usepackage[utf8]{inputenc} % UTF-8
\usepackage {tikz}
\usepackage{amsmath} % Graphs
\usepackage{etoolbox} % Well formatted Links in bib
\apptocmd{\thebibliography}{\raggedright}{}{} % Well formatted Links in bib
\usepackage{hyperref} % Links

%%% Title
\pretitle{
    \begin{center}
        \includegraphics[width=8cm]{../../resources/Haw_logo.png}
    \end{center}
}
\title{
    \begin{center}
        \Huge \textbf{Intelligente Systeme}\\
        {\color{blue}- Ausarbeitung -}\\~\\
        \Large Prüfungsform: Referat
    \end{center}
}
\author{Adrian Helberg\\ Matr.Nr. 2309051}
\date{Abgabe: 18.02.2020}

% Document
\begin{document}

    \maketitle

    \begin{abstract}
        Das Referat beschäftigt sich mit Konzepten der drei Kernbereiche aus \textit{intelligente
        Systeme}: \textbf{Suchen}, \textbf{Lernen} und \textbf{Verarbeitung von Sequenzen}. Zum
        einen werden Problemstellungen aufgezeigt, zum anderen auf Implementationen verwiesen, die
        in der zugehörigen Präsentation vorgestellt werden.\\ Eine
        Lösungsstrategie für das \textit{Problem des Handlungsreisenden (Traveling Salesman
        Problem)} wird im Kapitel \textbf{Suchen} beschrieben~\cite{suchen, salesman, impl}.
        \textit{Selbstorganisierte Karten (Selforganizing Maps)} zeigen eine Strategie des
        \textbf{Lernen}s~\cite{lernen, som} und ein \textit{Deep Learning}- Ansatz zur Datenanalyse
        von Sensordaten mobiler Endgeräte gibt Einblicke in die
        \textbf{Sequenzverarbeitung}~\cite{sequenzen}.\\
        Weiter lädt eine  Diskussion zu einem \textit{Trolley Problem} in Verbindung mit
        \textit{intelligenten Systemen} zu einer kritischen Fragestellung der \textbf{Ethik}
        ein~\cite{trolley} und zum Abschluss gibt es eine persönliche Einschätzung zur
        \textbf{Verantwortung als Informatiker}.
    \end{abstract}

    \tableofcontents
    \newpage

    \section{Stichwortverzeichnis}

    \subsection{Suchen}

    \begin{itemize}
        \item Problem des Handlungsreisenden (Traveling Salesman Problem)
        \item Informierte Suche
        \item Kombinatorisches Optimierungsproblem
        \item Evolutionärer Algorithmus (Genetischer Algorithmus)
        \begin{itemize}
            \item Chromosom
            \item Fitness
            \item Selektion
            \item Kreuzung
            \item Mutation
            \item Austausch
        \end{itemize}
        \item Visualisierung mit \textit{Java}
    \end{itemize}

    \subsection{Lernen}

    \begin{itemize}
        \item Unüberwachtes Lernen (unsupervised learning)
        \item Neuroinformatik
        \item Selbstorganisierende Karten (Self-organizing Maps)
        \begin{itemize}
            \item Klassifizierung
            \item Neuronale Nachbarschaft
            \item Topologieerhaltene Abbildung auf weniger Dimensionen
            \item \textit{Kohonen}-Netz
            \item Siegerneuronen
            \item Distanzfunktionen
        \end{itemize}
        \item Visualisierung anhand eines Beispiels
    \end{itemize}

    \subsection{Verarbeitung von Sequenzen}

    \begin{itemize}
        \item Sensordaten mobiler Endgeräte (Trägheitssensoren, Gyroskope)
        \item "`Tiefes"' Lernen (Deep Learning)
        \item Klassifizierungsmethode
        \item Analyse großer Datenmengen (\textit{Big Data})
        \item \textit{Human Activity Recognition} (\textit{HAR})
        \item Problem durch geringe Verfügbarkeit mobiler Ressourcen
        \item Herausarbeiten von Eigenschaften (Feature Detection)
        \item "`Shallow Deep Learning"'
        \begin{itemize}
            \item Segmente extrahieren
            \item Spektogramme
            \item Deep Learnung Modul
            \item Shallow Features
            \item Klassifizierung
            \item Training
            \item Mögliche Evaluation
        \end{itemize}
        \item Visualisierung anhand eines Beispiels
    \end{itemize}
    \newpage

    \section{Ethik}

    \subsection{Einleitung}

    Die Abwägung kritischer Verkehrsituationen ist zurzeit nur sehr begrenzt möglich. Zum einen hat
    der menschliche Fahrer meist nur wenige Sekunden, um eine Entscheidung zu treffen, zum anderen
    ist es in der Gefahrensituation kaum möglich die ganze Situation zu begreifen. In
    voraussichtlicher Zeit werden Autos durch genügend Rechenleistung in der Lage sein, die Folgen
    eines Verkehrsunfalls genauer analysieren zu können, als ein menschlicher Fahrer.
    Der Artikel "`Einer muss sterben - nur wer?"'~\cite{ethik} setzt sich mit dem moralischen
    Dilemma des autonomen Fahrens auseinander - ein \textit{Trolley Problem}. Das Problem beschreibt
    die Abwägung zwischen Menschenleben bei Verkehrsunfällen.

    \subsection{Kriterien}

    Nach welchen Kriterien ein Mensch in Gefahrensituationen im Straßenverkehr entscheidet, ist
    schwer zu greifen, da hierzu eine subjektive Entscheidung getroffen wird. Weiter gibt
    es eine Diskrepanz zwischem äußerem beabsichtigtem und tatsächlichem menschlichen Handeln, da
    moralische Grundsätze über die "`Gesellschaftliche Intelligenz"' vermittelt werden, und es daher
    Unterschiede im moralischen Denken der einzelnen Person gibt. Ein internationales
    Forscherteam konnte bisher über 40 Millionen Datensätze über getroffene Entscheidung von
    simulierten Verkehrssituationen sammeln, um diese Kriterien herauszuarbeiten.
    \newpage

    \subsection{Experiment}
    Über ein frei zugängliche Online-Plattform "`Moral Machine"' \\(\url{http://moralmachine.mit
    .edu/}) können die Nutzer verschiedene Verkehrssituationen "`durchspielen"' und so eine
    Entscheidung
    abgeben. Die Daten zeigen mehrere Tendenzen:
    \begin{itemize}
        \item Möglichst viele Leben retten
        \item Mehr Frauen, als Männer und mehr Kinder, als Alte werden gerettet
        \item Menschen mit höherer sozialer Stellung werden eher gerettet, als niedrigere
        \item Menschen, die bei Rot über die Straße gehen, werden weniger häufig gerettet
        \item Mitfahrer im Auto werden nicht häufiger gerettet, als Fußgänger
    \end{itemize}
    Weiter stellt sich heraus, dass es viele regionale Unterschiede gibt. In vielen asiatischen
    Ländern
    wurden junge Menschen beispielsweise seltener verschont als Alte.

    \subsection{Hersteller}
    Da das autonom agierende Fahrzeug die moralischen Entscheidung der Gesellschaft umsetzen muss,
    und die Programmierer dieses auch leisten müssen, ist es von hoher Wichtigkeit für diese
    Fragestellung, dass sich die Hersteller der Fahrzeuge äußern und
    aktiver Teil der Diskussion sind. Dennoch weichen Hersteller derzeit noch diesem moralischen
    Dilemma aus, da sie nach eigener Aussage zu sehr damit beschäftigt sind ein selbst fahrendes
    Auto zur Marktreife zu bringen.

    \subsection{Fazit}
    Die Schlussfolgerung des Artikels besagt, dass moralische Fragen in allen Bereichen der
    künstlichen Intelligenz gesamtgesellschaftlich diskutiert und erarbeitet werden müssen. Die
    Autorin sagt außerdem, dass sich ethische Zulässigkeiten von Normen nicht aus
    solchen nicht-repräsentativen Experimenten ableiten lassen. Damit vertritt sie eine ähnliche
    Meinung wie die deutsche Ethik-Kommision (\textit{"`Autonomes und vernetztes
    Fahren"'}~\cite{fahren}).

    \newpage

    \subsection{Kritik}
    Dass sich Hersteller von autonomen Fahrzeugen derzeit noch aus der Diskussion über dieses
    moralische Dilemma heraushalten, sehe ich sehr kritisch. Meiner Meinung nach stützt sich die
    Industrie nur auf den statistischen Erfolg, den autonomes Fahren in Zukunft bringen kann.
    Deutlich wird dies an einem im Artikel genanntes Argument der Autoindustrie:\\~\\
    \textit{"`Ist der Straßenverkehr erst in Roboterhand, sollten ohnehin keine Unfälle mehr
    passieren"'}\\~\\
    Es wird möglich sein die Gesamtanzahl an Verkehrunfällen drastisch zu reduzieren, wobei das
    Thema der Ethik allerdings zurückbleibt. Moralische Grundsätze von allen Seiten zu diskutieren
    sollte ein fest integriertes System sein, das mit der Gesellschaft "`mitwächst"'.

    \section{Verantwortung als Informatiker}
    Da ich mich im Rahmen meiner Bachelorthesis ebenfalls mit Kernbereichen aus \textit{intelligente
    Systeme} beschäftigen möchte, sehe ich mich in Zukunft unmittelbar in einer Position die Welt
    "`mitformen"' zu können. Debei sollten intelligente Systeme dem Menschen unterstützend zur
    Seite stehen, ohne selbst weitreichende Entscheidungen zu treffen. Da ich mich persönlich
    vor der Bearbeitung dieses Referats nur sehr wenig mit der Ethik in der Informatik beschäftigt
    habe, kann ich nachvollziehen, warum die Industrie beispielsweise zurzeit eher abgeneigt ist,
    sich mit dieser zu beschäftigen.\\ Dennoch sollte für dieses Thema sensibilisiert werden, um so
    einen sich entwickelnden Prozess zur Diskussion von ethischen Fragen zu etablieren.\\~\\
    Da sich gesellschaftliche Normen in verschiedenen Gesellschaften stark unterscheiden können,
    muss ich mir als Informatiker, im speziellen als Programmierer, darüber im Klaren sein, dass
    ich evtl. "`ethische Funktionalitäten"' imlpementieren muss, welche ich nicht selbst vertrete.
    Weiter möchte ich nicht in einer Welt leben, in der durch den Erwerb eines autonom fahrenden
    Autos eine "`ethische Überlegenheit"' miterworben wird, bei der immer der Besitzer des Autos
    geschützt wird, da er für dieses gezahlt hat.

    \bibliographystyle{plain}
    \bibliography{script}
\end{document}